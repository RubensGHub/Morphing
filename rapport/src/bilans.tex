\section{Bilans personnels}
\subsection{Bilan Rubens}

Ce projet m'a permis de beaucoup gagner en expérience sur de nombreux aspects, notamment concernant la création d'application Javafx, ainsi que l'arrangement du code avec la méthode PAC. J'ai eu l'impression de bien plus en apprendre sur Java en général avec un réel projet qui permettait de mettre en application toutes les connaissances (mélangé avec des mathématiques), plutôt qu'avec les différents TPs. Également, travailler avec des gens que l'on connaît peu est une expérience dont j'avais peu l'habitude et qui sera importante plus tard, donc je pense que c'était une bonne idée, bien que déplaisante au départ. 
Finalement, je trouve que ce projet s'est révélé très intéressant, enrichissant et amusant.
\subsection{Bilan Alex}

Ce projet fut l'occasion de mettre en pratique nos compétences acquises tout au long de cette première année en cycle ingénieur. J'ai pu comprendre profondément l'utilité de l'étape de conception d'un projet avec les différents diagrammes à concevoir pour ne pas se perdre dans le code. De plus, j'ai appris à utiliser GitHub qui est un outil dont on m'avait déjà beaucoup parlé mais que je n'ai jamais utilisé. Sa prise en main a été assez difficile pour moi au début, car j'aime comprendre exactement ce que je fais et ne connaissant pas cette application, j'ai dû me lancer dans un projet sans connaître parfaitement la prise en main de GitHub.

Néanmois, la partie la plus importante et intéressante selon moi est l'organisation d'un groupe de travail. En effet, durant mes deux années de classe préparatoire aux grandes écoles, le travail de groupe n'était pas du tout mit en avant. Ainsi, avec ce projet, j'ai pu voir la difficulté de travailler avec des personnes qui n'ont pas les mêmes idées, les mêmes manières de penser, les mêmes manières de coder... J'ai donc beaucoup appris sur cet aspect là et j'en garde une très belle expérience.


\subsection{Bilan Paul}
Le projet de morphing a été une expérience extrêmement enrichissante pour moi. J'ai pu approfondir mes connaissances en algorithmes, notamment en travaillent sur l'algorithme de Beier-Neely pour faire du morphing d'images et des transformations géométriques, tout en mettant en pratique des concepts théoriques complexes. Travailler en groupe a également été très bénéfique; j'ai appris différentes méthodes de programmation grâce aux contributions variées de mes coéquipiers, ce qui a enrichi mes compétences et ma perspective sur la résolution de problèmes.

\subsection{Bilan Romain}
J'ai trouvé ce projet très intéressant. Cela m'a permis de mieux comprendre le langage Java ainsi que la partie JavaFx. En effet, nous avons eu des tp sur ceci, mais là nous avons eu la possibilité de voir comment utiliser toutes les fonctionnalités vues en tp dans un projet concret. L'utilisation de la méthode PAC a été plus claire grâce à ce projet. Effectivement, je n'avais pas très bien compris comment celle-ci fonctionnait et grâce à notre travail j'ai finalement compris pourquoi l'utiliser et comment cela fonctionnait. Enfin, le fait de ne pas avoir pu choisir les personnes avec qui nous voulions travailler fut une belle expérience. En effet, je n'en avais pas l'habitude jusque-là et j'ai dû faire face à d'autres manières de penser, d'autres manières de travailler. Cette expérience me sera, j'en suis sur, très bénéfique pour la suite de ma vie étudiante et professionnelle.
\subsection{Bilan Ryan}
Ce projet a été l'occasion pour moi de constater une énième fois les nécessités impérieuses auxquelles je dois faire face, tant dans la gestion d'un groupe, que dans la gestion de mon temps. En effet, j'ai pu constater que la gestion d'un groupe de travail est une tâche ardue, qui nécessite une communication constante, adapté à l'auditoire et, malheureusement, assez répétitive. Autrement, j'ai pu constater que la gestion de mon temps est un aspect que je dois améliorer, car j'ai souvent été attentiste dans les situations requérant la finalisation du travail d'autrui, ce qui ne m'a pas permis d'aboutir le projet en totalité. 