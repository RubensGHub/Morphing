\section{Introduction}

\subsection{Contexte}

Le morphing est une technique d'animation et d'effets visuels utilisée pour créer une transformation fluide entre deux images ou objets distincts. Ce processus permet de voir une image se métamorphoser progressivement en une autre, en donnant l'illusion que la première se transforme directement en la seconde.

Le morphing est couramment utilisé dans diverses applications, notamment :

\begin{description}
    \item[Cinéma et Télévision] Pour créer des effets spéciaux impressionnants où un personnage ou un objet change de forme de manière spectaculaire ;
    \item[Vidéo musicale] Les clips musicaux utilisent souvent le morphing pour des transitions visuelles créatives et artistiques ;
    \item[Publicité] Pour attirer l'attention et illustrer des transformations de produits ou de services ;
    \item[Logiciels de photographie et d'animation] Offrent des outils de morphing pour les artistes et les animateurs afin de créer des effets visuels captivants.
\end{description}

Un exemple célèbre de morphing dans le cinéma est celui utilisé dans le film "Terminator 2: Judgment Day" (1991), où le personnage du T-1000 change de forme de manière impressionnante. Cette technique a depuis été perfectionnée et est devenue un outil standard dans les effets visuels.


\subsection{Objectif}

L'objectif de ce projet est de réaliser une application de morphing permettant de créer une animation d'une image  \emph{source} à uneimage de  \emph{destination}. Pour ce faire, trois exercices différents sont proposés par le sujet :

\begin{enumerate}
    \item \textbf{Formes simples} : deux polygones de même couleur ;
    \item \textbf{Formes courbées} : deux formes quelconques avec des courbures ;
    \item \textbf{Visages} : deux visages.
\end{enumerate}

