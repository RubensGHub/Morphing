\section{Cas des polygones simples}
\label{sec:polygones-simples}
En premier lieu, nous allons nous intéresser au cas des polygones simples.
Ce faisant, ceci sera l'occasion pour nous de faire notre premier pas dans l'\emph{informatique graphique},
en traitant des cas élémentaires du problème de la morphose.
\subsection{Généralités} 
\label{sec:generalites}
\paragraph{Conventions}
Pour toute la suite, et sauf mention contraire, on se place dans le $\R$-evn $\R^2$ muni de la norme euclidienne.
On notera pour tout vecteur $x\in\R^2$, $X\in \mathcal{M}_{2,1}(\R)$ la matrice colonne associée à $x$ dans la base canonique de $\R^2$.

\begin{definition}\label[definition]{def:1}
    Soit $n$ un entier naturel non nul. On appelle \emph{polygone simple} de $\R^2$ tout $n$\_uplet $P=(p_1,\ldots,p_n)$ de $\R^2$ tels que:
    \begin{enumerate}
        \item Les segments ne se croisent pas, c'est-à-dire que pour chaque paire de segments \( [p_i, p_{i+1}] \) et \( [p_j, p_{j+1}] \) (où \( p_{n+i} \) est \( p_i \)), les segments ne partagent pas de points autres que les sommets.
        \item Chaque sommet $p_i$ est partagé par exactement deux segments.
    \end{enumerate}
    On notera $p_{i^+}$ le segment $[p_i, p_{i+1}]$ et $p_{i^-}$ le segment $[p_i, p_{i-1}]$. Enfin, on notera $\mathbb{P}$ l'ensemble des polygones simples de $\R^2$.
\end{definition}

\begin{definition}\label[definition]{def:2}
    Soit $P\in\mathbb{P}$ un polygone simple de $\R^2$.
    \begin{enumerate}
        \item On appelle \emph{ordre} de $P$ le nombre $n$ de composantes de $P$.
        \item On appelle \emph{arête} de $P$ tout segment $p_{i^+}$ ou $p_{i^-}$ pour $i\in\{1,\ldots,n\}$. 
        \item On appelle \emph{sommet} de $P$ tout vecteur $p_i$ pour $i\in\{1,\ldots,n\}$.
    \end{enumerate}
    Subséquemment, pour $n$ un entier naturel non nul, on note $\mathbb{P}_n$ l'ensemble des polygones simples de $\R^2$ d'ordre $n$.

\end{definition}

\begin{definition}\label[definition]{def:3}
    Soit $P\in\mathbb{P}$ un polygone simple de $\R^2$.
    On appelle \emph{intérieur} de $P$, noté $\overset{\circ}{P}$ , l'ensemble des points $x\in\R^2$ tels que $x$ est à gauche de chaque arête de $P$.
\end{definition}




\paragraph{Situation} À ce stade, l'enjeu de cette première partie est de déterminer un algorithme permettant 
la morphose d'un polygone simple $P$ vers un autre polygone simple $Q$. Pour ce faire, il nous faut nous intéresser
aux conditions d'une telle transformations, ainsi qu'à sa réalisation.

\subsection[Morphing naif]{Un premier algorithme de morphing}
\label{sec:morphing-naif}
\subsubsection{Principe et notation}
\paragraph{Principe} L'idée de cet algorithme est de déformer progressivement le polygone $P$ en un autre polygone $Q$.
Pour ce faire, une première approche consiste à déformer chaque sommet $p_i$ de $P$ en un sommet $q_i$ de $Q$ par
une interpolation linéaire. Ainsi, on obtient une suite de polygones $(P_{k})_{0\leq k\leq N}$ 
où $N$ est le nombre d'images intermédiaires voulues et tels que $P_{0}=P$, $P_{N}=Q$.

\begin{coder}
    Pour que l'algorithme soit correct, il est nécessaire que les polygones $P$ et $Q$ soient de même ordre.
\end{coder}

\paragraph*{Notation} Soit $n,N>0$ et $(P_k)_{0\leq k\leq N}$ une suite de polygones de $(\mathbb{P}_{n})^{\NN}$.
 On notera $p^{(k)}_1,\ldots,p^{(k)}_n$ les sommets de $P_k$.

\paragraph{} Ce faisant, pour le calcul des images intermédiaires on donne l'algorithme suivant:\\[0.5cm]
\begin{algorithm}[H]
    \caption{générationFramesNaif}\label{alg:0}
    \SetAlgoLined
    \KwData{$P,Q\in\mathbb{P}_n$ deux polygones, $N>0$ le nombre de frames}
    \KwResult{Une suite de polygones $(P_k)$}
    $P^*\gets(P_1,\dots,P_N)$\\
    \For{$k\gets 0$ \KwTo $N$}{
        $t\gets\frac{k}{N}$\\
        $P_k=(p^{(k)}_1,\ldots,p^{(k)}_n)$\\ 
        \For{$i\gets 1$ \KwTo $n$}{
            $p^{(k)}_{i}\gets (1-t)*p_i+t*q_i$
        }
    }
    \Return $P^*$
\end{algorithm}

\subsubsection{Cas des images matricielles}
\paragraph{Principe} Dans le cadre de l'exercice, la donnée du problème est consituée de deux images matricielles $P$ et $Q$
de taille $L\times l$ représentant respectivement des polygones simples de couleur même couleur. Naturellement,
plusieurs méthodes algorithmiques peuvent être utilisées pour encoder l'information géométrique portée par une image.
Entre autre, des algorithme de \emph{détection de contours} ou de \emph{segmentation} peuvent être utilisés. Toutefois, et par soucis
de simplicité, nous allons considérer que l'information géométrique est directement encodée par l'utilisateur lors de la saisie des
\emph{points de contrôle}. De plus, pour chaque pixel de coordonnées $(i,j)$, on a :

$$
\begin{cases}
    \1_{P}(i,j) \= 1 & \text{si } (i,j)\in\overset{\circ}{P}\\
    \1_{P}(i,j) \= 0 & \text{sinon}
\end{cases}
$$
\begin{coder}
    Pour que l'algorithme soit correct, il est nécessaire que les images $P$ et $Q$ soient de même taille. De plus,
    si les points de contrôle saisis par l'utilisateur ne correspondent pas aux sommets des polygones,
    les résultats de l'algorithme peuvent être inattendus.
\end{coder}

\begin{propriete}\label[proprieté]{prop:1}
    Soit $u,v\in\R^2$ deux vecteurs.
    Alors $u$ est à gauche de $v$ si et seulement si $det(u,v)>0$.
\end{propriete}
\begin{dem}
    Admis.
\end{dem}


\begin{propriete}\label[proprieté]{prop:1}
    Soit $p=[u,v], (u,v)\in\R^2$ un segment et $x\in\mathbb{R}^2$ un vecteur.
    Alors $x$ est à gauche de $p$ si et seulement si $det(p,[u,x])>0$.
\end{propriete}
\begin{dem}\label{dem:1}
    Soit $u,v\in\R^2$ deux vecteurs. On note $\theta$ l'angle orienté entre $u$ et $v$.
    Alors, 
    \begin{align}
        u \text{ est à gauche de } v &\iff \text{det}(u,v)>0\\
        &\iff \left\lVert u \right\rVert \cdot \left\lVert v \right\rVert \sin(\theta)>0\\
        &\iff \sin(\theta)>0\\
        &\iff \theta\in\left]0,\pi\right[\\
        &\iff u \text{ est dans le demi-espace à gauche de } v
    \end{align}
\end{dem}
\paragraph{} Ce faisant, pour le calcul des images intermédiaires, 
et la determination de l'intérieur du polygone, de on donne la suivante:\\[0.5cm]
%% Détermination de l'intérieur d'un polygone

\begin{algorithm}[H]
    \SetAlgoLined
    \KwData{Une liste de points $tab$ représentant le polygone, un point $P$}
    \KwResult{Un booléen indiquant si le point $P$ est à l'intérieur du polygone}
    
    $nbp \gets \text{taille de } tab$ 
    \For{$i \gets 0$ \KwTo $nbp-1$}{
        $A \gets tab[i]$ 
        $B \gets tab[(i+1) \mod nbp]$ 
        
        $D \gets (B.x - A.x, B.y - A.y)$ 
        $T \gets (P.x - A.x, P.y - A.y)$ 
        
        $d \gets det(D,T)$ 
        
        \If{$d < 0$}{
            \Return Faux 
        }
    }
    \Return Vrai;
    
    \caption{isInside}
\end{algorithm}

\paragraph{} On en déduit l'algorithme naif pour le mophing de formes unies simples:\\[0.5cm]

\begin{algorithm}[H]
    \SetAlgoLined
    \KwData{Deux images matricielles $P$ et $Q$ de taille $L\times l$, un entier $N$}
    \KwResult{Une suite d'images matricielles $P^*$}
    
    $nbp \gets \text{taille de } tab$\\
    $P^* \gets \text{générationFramesNaif}(P,Q)$\\
    \For{$i \gets 0$ \KwTo $nbp-1$}{
        \For{$x \gets 0$ \KwTo $L-1$}{
            \For{$y \gets 0$ \KwTo $l-1$}{
                if($isInside(P^*_i, (x,y))$){
                    $P^*_i[x][y] \gets color$
                }
            }
        }

    }
    \Return $P^*$
    
    \caption{morphingNaif}
\end{algorithm}
\newpage
\paragraph{Résultats} Considérons une exécution de l'algorithmes sur deux instances du problème.\\
\begin{figure}[h!]
        \centering
        \begin{subfigure}{0.4\textwidth}
            \centering
            \includegraphics[width=0.8\textwidth]{img/p1/1.png}
            \caption{Image initiale}
        \end{subfigure}
        \begin{subfigure}{0.4\textwidth}
            \centering
            \includegraphics[width=0.8\textwidth]{img/p1/2.png}
            \caption{Image intermédiaire 1}
        \end{subfigure}
        \begin{subfigure}{0.4\textwidth}
            \centering
            \includegraphics[width=0.8\textwidth]{img/p1/3.png}
            \caption{Image intermédiaire 2}
        \end{subfigure}
        \begin{subfigure}{0.4\textwidth}
            \centering
            \includegraphics[width=0.8\textwidth]{img/p1/4.png}
            \caption{Image intermédiaire 3}
        \end{subfigure}
        \begin{subfigure}{0.4\textwidth}
            \centering
            \includegraphics[width=0.8\textwidth]{img/p1/5.png}
            \caption{Image intermédiaire 4}
        \end{subfigure}
        \begin{subfigure}{0.4\textwidth}
            \centering
            \includegraphics[width=0.8\textwidth]{img/p1/6.png}
            \caption{Image finale}
        \end{subfigure}
        \caption{Résultats de l'algorithme de morphing}
\end{figure}

\begin{codeb}
    Dans le cas de polygones simples, relativements semblables dans leur géométrie, 
    l'algorithme de morphing naif donne des résultats satisfaisants.
\end{codeb}

\paragraph{En pratique} Toutefois, il apparaît qu'une telle approche ne préserve pas les propriétés géométriques
du polygone. En effet, les images intermédiaires peuvent contenir des segments qui se croisent, 
ou des sommets qui ne sont pas partagés par exactement deux segments. Dans un tel cas, la transformations
ne paraît pas naturelle.

Ainsi, il est nécessaire de trouver une approche plus rigoureuse pour le morphing de polygones simples.

\subsection[Méthode L-A]{Morphing par interpolation longeur-angle}

