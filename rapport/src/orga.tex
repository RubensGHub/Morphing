\section{Organisation au sein du groupe}

Pour ce projet, nous avons été réparti en groupe de cinq étudiants (GI). Il faut donc s'organiser pour mener à bien le projet pour le 31 mai 2024, date à la quelle la soutenance est prévue.


\subsection{Espace de travail : GitHub}

Ce projet a été l'occasion de découvrir GitHub, un logiciel de partage et de stockage de fichiers optimisé et pensé pour le développement informatique. Il permet de s'échanger des fichiers à distance, voir l'historique des modificaitons et de gérer les conflits en cas de fichiers modifiés par deux presionnes différentes en même temps. C'est pourquoi nous avons choisi cet outil pour nous aider dans la réalistion de notre application de morphing.


\subsection{Répartition des tâches}

Une fois l'environnement de travail pris en main, il nous a fallu nous répartir des tâches à réaliser.

D'abord, il nous fallait nous mettre d'accord sur la méthode à suivre. POur cela, tout le monde à fait des recherches de son côté durant la première semaine en notant les sources consultées qui seront notées dans la bibliographie de ce rapport. Après plusiseurs recherches, nous avons opté pour l'implémentation de la méthode des lignes plutôt que la triangulation de Delaunay.

Globalement, deux groupes se sont formés. Romain et Rubens ce sont occupés de la partie IHM de l'applicaiton ainsi que de la mise en place des contrôleurs. Quant à Rayan et Paul, ils sont focalisés sur la partie back-end du Java avec l'implémentation des algorithmes et des fontions mathématiques nécessaires. Alexandre, a faisait la liaison entre les deux pôles. Néanmoins, quand l'un d'entre nous a une question, il est possible de s'écahnger certaines taĉhes pour s'entraîder. Notre organisaiton est plutôt flexible.



