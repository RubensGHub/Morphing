\section{Morphing de formes courbes}
\label{sec:morphing_courbes}

\paragraph{} Les polylignes à courbure nulle ne permettent pas une représentation précise de formes courbes. 
Pour pallier ce problème, nous allons nous appuyer sur les \emph{courbes B-splines}, couramment utilisées en CAO pour représenter des formes courbes \cite{pansu2004bsplines}.


\subsection{Propédeutique au morphing de courbes splines}

\subsubsection{Splines}

\begin{definition}
    Considérons des réels $u_0 < u_1 < \ldots < u_m$, et $p\in\NN$.
    On définit les \emph{fonction B-Splines} $B_{i, p}$ par récurrence sur $p$ et $i$ dans $\NN$ comme suit :
    
    \begin{equation}
        \begin{cases}
            \text{Pour } 0 \leq i \leq m-1 \\
            B_{i, 0}(u)=1 \text{ si } u \in[u_i, u_{i+1}[, \quad B_{i, 0}(u)=0 \text{ sinon }
        \end{cases}
    \end{equation}
    \begin{equation}
        \begin{cases}
            \text{Pour } 0 \leq i \leq m-1 \\
            B_{i, p}(u)=\frac{u-u_i}{t_{i+p}-t_i}B_{i, p-1}(u)+\frac{u_{i+p+1}-u}{u_{i+p+1}-u_{i+1}}B_{i+1, p-1}(u)
        \end{cases}
    \end{equation}
    
\end{definition}
\paragraph{Notation.} Soit $j=1, \ldots, m+1-i$. , on note:
$
\begin{cases}
    \omega_{i, j}(u)=\frac{u-u_i}{u_{i+j}-u_i} & \text{si } u_i<u_{i+1},\\
    \omega_{i, j}=0 & \text{sinon}.
\end{cases}$
\vspace*{0.5cm}

\begin{coder}
\textbf{Convention.  } Pour la suite, toute fonction dont le dénominateur est nul sera considérée comme nulle.
\end{coder}

\begin{definition}
    Ainsi, on a pour $i \in \{0, \ldots, m-p-1\}$ et $p \in \NN$ :
    \begin{align*}
        B_{i, 0}(u)=&1 \quad \text { pour } \quad t \in\left[u_i, t_{i+1}[=0\text {, }\right.\\
        B_{i, p}(u)=&\sum_{j=1}^{m+1-i} \omega_{i, j}(u)B_{i, p-1}(u) \quad \text { pour } p>0.
    \end{align*}
\end{definition}

\subsubsection{Courbes B-Splines}

\begin{definition}
    Soit $m\in\NN$. On appelle $(u_i)_{0 \leq i \leq m}$, \emph{vecteur de noeuds}, et $p$, \emph{degré de la B-spline}.
    On considère aussi des \emph{points de contrôle} $\mathbf{P}_1, \dots, \mathbf{P}_m$ de $\R^n$. De fait, $(\mathbf{P}_i)_{0 \leq i \leq m}$ forme un \emph{polygone de contrôle}.
    La \emph{courbe B-Spline} d'ordre $p$ associée à ces données est définie par:
    \begin{equation}
        u \longmapsto C(u)=\sum_{i=0}^{m} B_{i, p}(u)\mathbf{P}_i.
    \end{equation}
\end{definition}

\begin{propriete}
    Supposons que \( C(u) \) soit une courbe B-spline de degré \( p \) définie comme suit :
    \[
    C(u) = \sum_{i=0}^{n} B_{i,p}(u) \mathbf{P}_i
    \]
    
    Soit le point de contrôle \( \mathbf{P}_i \) déplacé vers une nouvelle position \( \mathbf{P}_i + \mathbf{v} \). Alors, la nouvelle courbe B-spline \( D(u) \) de degré \( p \) est la suivante \cite{pansu2004bsplines} :
    \begin{equation}
        D(u) = C(u) + N_{i,p}(u)\mathbf{v}
    \end{equation}
\end{propriete}
\begin{dem}
    \begin{align*}
        D(u) &= \sum_{i=0}^{n} B_{i,p}(u) (\mathbf{P}_i + \mathbf{v}) \\
        &= \sum_{i=0}^{n} B_{i,p}(u) \mathbf{P}_i + \sum_{i=0}^{n} B_{i,p}(u) \mathbf{v} \\
        &= C(u) + \sum_{i=0}^{n} B_{i,p}(u) \mathbf{v} \\
        &= C(u) + N_{i,p}(u) \mathbf{v}
    \end{align*}
\end{dem}
\paragraph{En pratique} Desormais, il nous est possible, sur la base de points de contrôle, d'un vecteur de noeud et du degré de la B-spline, de générer une courbe B-spline
comme ci-dessous:

\begin{figure}[H]
    \begin{tikzpicture}
        \Bezier{0,0}{1,1}{2,-1}{3,0}
        \begin{scope}[xshift=4cm]
            \Bezier{0,0}{9,2}{-2,2}{7,0}
        \end{scope}
        \begin{scope}[yshift=-5cm]
            \Bezier{0,0}{1,3}{2,3}{7,0}
    \end{scope}
    \begin{scope}[xshift=8cm,yshift=-5cm]
        \Bezier{0,0}{-2,4}{4,-1}{5,0}
    \end{scope}
\end{tikzpicture}
\caption{Courbes B-splines}
\label[figure]{fig:bsplines}
\end{figure}
\begin{lemme}
    \label{lem:bspline_fermee}
    Soit $\mathcal{C}=(U,P,d)$ une courbe B-Spline.
    Alors, $\mathcal{C}$ est une courbe fermée si et seulement si :
    \begin{enumerate}
        \item $\mathbf{P}_0=\mathbf{P}_m$, où $m=ord(P)$,
        \item $U$ est un vecteur de noeuds uniforme.
    \end{enumerate}
\end{lemme}

\begin{corollaire}
    \label{cor:1}
    Soient $m, N>0$ et $v=(\mathbf{v}_{i,k})_{(i,k)\in\intervalle{0}{m}\times\intervalle{0}{N}}\in\mathcal{M}_{m+1,N+1}(\R^2)$\\
    Soit $\mathcal{C}_0=(U,P^{(0)},d)$ et $\mathcal{C}_k=(U,P^{(k)},d)$, $k\in\NN$, deux courbes B-Splines telles que :
    \begin{enumerate}
        \item $\forall k\in\intervalle{0}{N},\forall i\in\intervalle{0}{m},\, \mathbf{P} _i^{(k)}=\mathbf{P} _i^{(0)}+\mathbf{v}_{i,k}$, 
        \item $U$ est un vecteur de noeuds uniforme.
    \end{enumerate}
\end{corollaire}

\newpage
\subsection{Morphing de courbes B-splines}
\paragraph{Principe} Notre approche se distingue en deux phases. Premièrement, laisser à l'utilisateur le soin de réaliser
l'ajout de points de contrôles, et subséquemment, l'ajustement de la courbe qui modélise la forme souhaitée. Ensuite, nous procédons
au calcul des fonctions de base $B_{i,p}(u)$ ainsi qu'aux points de la courbes $C(u)$ pour l'image source. Ce faisant, en réalisant
une interpolation linéaire des points de controles entre les polygones de contrôles de l'image source et de l'image destination,
nous sommes en mesure de déterminer pour chaque image intermédiaire $0\leq k\leq N$ la position $C^{(k)}(u)$ de la courbe en utilisant
le corollaire \ref{cor:1}.
\subsubsection{Calcul d'une courbe B-spline}
\paragraph{Calcul de la courbe} Dans le sous-domaine mathématique de l'analyse numérique, l'algorithme de de Boor \cite{sheneGeometryNotes} est un algorithme polynomial 
et numériquement stable pour évaluer les courbes splines sous forme de B-splines. 
Il s'agit d'une généralisation de l'algorithme de de Casteljau pour les courbes de Bézier. 
Cet algorithme a été conçu par le mathématicien germano-américain Carl R. de Boor. 
Des variantes simplifiées et potentiellement plus rapides de l'algorithme de de Boor ont été créées, mais elles souffrent d'une stabilité comparativement plus faible.
\begin{algorithm}[h!]
    \BlankLine
    \SetKwFunction{FMain}{calculerPoint}
    \SetKwProg{Fn}{Fonction}{:}{}
    \Fn{\FMain{$u$, $controlPoints$, $nodeVector$}}{
        $k \leftarrow trouverIntervalle(u, nodeVector)$\;
        $d \leftarrow controlPoints.subList(k - deg, k + 1)$\;
        
        \BlankLine
        \tcp{Algorithme de De Boor}
        \For{$r \leftarrow 1$ \KwTo $deg$}{
            \For{$j \leftarrow k$ \KwTo $k - r$ \KwBy $-1$}{
                $i \leftarrow j - k + deg$\;
                $a \leftarrow \frac{u - nodeVector.get(j)}{nodeVector.get(j + deg - r + 1) - nodeVector.get(j)}$\;
                $interpolated \leftarrow d.get(i - 1).nextPoint(d.get(i), a)$\;
                $d.set(i, interpolated)$\;
            }
        }
        
        \BlankLine
        \Return{$d.get(deg)$}\;
    } 
    \caption{Calcul courbes B-splines}
    \label{algo:deboor}
\end{algorithm}
\paragraph{Remarque. } On donne le pseudo-code de la fonction \texttt{trouverIntervalle} dans l'algorithme \ref{algo:trouver-intervalle} en annexe.

\paragraph{Algorithme} On donne ci-après l'algorithme principal pour le morphing de courbes B-splines, basé sur l'approche évoquée plus haut. Malheureusement,
faute d'appréciation, nous n'avons pas pu tester l'algorithme. Nous en donnons une implémentation en Java complète dans les sources du projet. Par ailleurs,
 la javadoc est disponible pour une meilleure compréhension de l'implémentation.
\begin{algorithm}[h!]
    \SetKwInOut{Input}{Entrée}
    \SetKwInOut{Output}{Sortie}
    
    \Input{Les points de contrôle de l'image source $sourcePoints$, les points de contrôle de l'image destination $destinationPoints$, le nombre d'images intermédiaires à générer $numFrames$}
    \Output{Une liste d'images intermédiaires représentant le morphing de courbes B-splines}
    
    \BlankLine
    \tcp{Vérifier que les listes de points de contrôle ont la même taille}
    \If{$sourcePoints.size() \neq destinationPoints.size()$}{
        \Return{$null$}\;
    }
    
    \BlankLine
    \tcp{Calculer les vecteurs de noeuds pour les courbes source et destination}
    $sourceNodeVector \leftarrow calculerNodeVector(sourcePoints.size(), deg)$\;
    $destinationNodeVector \leftarrow calculerNodeVector(destinationPoints.size(), deg)$\;
    
    \BlankLine
    \tcp{Générer les images intermédiaires}
    $intermediateImages \leftarrow$ une liste vide\;
    \For{$i \leftarrow 0$ \KwTo $numFrames$}{
        $t \leftarrow \frac{i}{numFrames}$\;
        
        \BlankLine
        \tcp{Calculer les points de contrôle intermédiaires par interpolation linéaire}
        $intermediatePoints \leftarrow$ une liste vide\;
        \For{$j \leftarrow 0$ \KwTo $sourcePoints.size()$}{
            $sourcePoint \leftarrow sourcePoints.get(j)$\;
            $destinationPoint \leftarrow destinationPoints.get(j)$\;
            $x \leftarrow (1 - t) \cdot sourcePoint.getX() + t \cdot destinationPoint.getX()$\;
            $y \leftarrow (1 - t) \cdot sourcePoint.getY() + t \cdot destinationPoint.getY()$\;
            $intermediatePoints.add(new Point(x, y))$\;
        }
        
        \BlankLine
        \tcp{Calculer les points de la courbe B-spline intermédiaire}
        $intermediateCurve \leftarrow$ une liste vide\;
        \For{$u \gets 0$ \KwTo $1$ \KwBy $0.01$}{
            $point \leftarrow calculerPoint(u, intermediatePoints, sourceNodeVector)$\;
            $intermediateCurve.add(point)$\;
        }
        
        \BlankLine
        \tcp{Ajouter l'image intermédiaire à la liste}
        $intermediateImage \leftarrow genererImage(intermediateCurve)$\;
        $intermediateImages.add(intermediateImage)$\;
    }
    
    \BlankLine
    \Return{$intermediateImages$}\;
    \caption{Morphing de courbes B-splines}
    \label{algo:mainalgorithm}
\end{algorithm}

