\section{Morphing de formes courbes}
\label{sec:morphing_courbes}

\paragraph{} Les polylignes à courbure nulle ne permettent pas une représentation précise de formes courbes. 
Pour pallier ce problème, nous allons nous appuyer sur les \emph{courbes B-splines}, couramment utilisées en CAO pour représenter des formes courbes \cite{pansu2004bsplines}.


\subsection{Propédeutique au morphing de courbes splines}

\subsubsection{Splines}

\begin{definition}
    Considérons des réels $u_0 < u_1 < \ldots < u_m$, et $p\in\NN$.
    On définit les \emph{fonction B-Splines} $B_{i, p}$ par récurrence sur $p$ et $i$ dans $\NN$ comme suit :
    
    \begin{equation}
        \begin{cases}
            \text{Pour } 0 \leq i \leq m-1 \\
            B_{i, 0}(u)=1 \text{ si } u \in[u_i, u_{i+1}[, \quad B_{i, 0}(u)=0 \text{ sinon }
        \end{cases}
    \end{equation}
    \begin{equation}
        \begin{cases}
            \text{Pour } 0 \leq i \leq m-1 \\
            B_{i, p}(u)=\frac{u-u_i}{t_{i+p}-t_i}B_{i, p-1}(u)+\frac{u_{i+p+1}-u}{u_{i+p+1}-u_{i+1}}B_{i+1, p-1}(u)
        \end{cases}
    \end{equation}
    
\end{definition}
\paragraph{Notation.} Soit $j=1, \ldots, m+1-i$. , on note:
$
\begin{cases}
    \omega_{i, j}(u)=\frac{u-u_i}{u_{i+j}-u_i} & \text{si } u_i<u_{i+1},\\
    \omega_{i, j}=0 & \text{sinon}.
\end{cases}$
\vspace*{0.5cm}

\begin{coder}
\textbf{Convention.  } Pour la suite, toute fonction dont le dénominateur est nul sera considérée comme nulle.
\end{coder}

\begin{definition}
    Ainsi, on a pour $i \in \{0, \ldots, m-p-1\}$ et $p \in \NN$ :
    \begin{align*}
        B_{i, 0}(u)=&1 \quad \text { pour } \quad t \in\left[u_i, t_{i+1}[=0\text {, }\right.\\
        B_{i, p}(u)=&\sum_{j=1}^{m+1-i} \omega_{i, j}(u)B_{i, p-1}(u) \quad \text { pour } p>0.
    \end{align*}
\end{definition}

\subsubsection{Courbes B-Splines}

\begin{definition}
    Soit $m\in\NN$. On appelle $(u_i)_{0 \leq i \leq m}$, \emph{vecteur de noeuds}, et $p$, \emph{degré de la B-spline}.
    On considère aussi des \emph{points de contrôle} $\mathbf{P}_1, \dots, \mathbf{P}_m$ de $\R^n$. De fait, $(\mathbf{P}_i)_{0 \leq i \leq m}$ forme un \emph{polygone de contrôle}.
    La \emph{courbe B-Spline} d'ordre $p$ associée à ces données est définie par:
    \begin{equation}
        u \longmapsto C(u)=\sum_{i=0}^{m} B_{i, p}(u)\mathbf{P}_i.
    \end{equation}
\end{definition}

\begin{propriete}
    Supposons que \( C(u) \) soit une courbe B-spline de degré \( p \) définie comme suit :
    \[
    C(u) = \sum_{i=0}^{n} B_{i,p}(u) \mathbf{P}_i
    \]
    
    Soit le point de contrôle \( \mathbf{P}_i \) déplacé vers une nouvelle position \( \mathbf{P}_i + \mathbf{v} \). Alors, la nouvelle courbe B-spline \( D(u) \) de degré \( p \) est la suivante \cite{pansu2004bsplines} :
    \begin{equation}
        D(u) = C(u) + N_{i,p}(u) \mathbf{v}
    \end{equation}
\end{propriete}
\begin{dem}
    \begin{align*}
        D(u) &= \sum_{i=0}^{n} B_{i,p}(u) (\mathbf{P}_i + \mathbf{v}) \\
        &= \sum_{i=0}^{n} B_{i,p}(u) \mathbf{P}_i + \sum_{i=0}^{n} B_{i,p}(u) \mathbf{v} \\
        &= C(u) + \sum_{i=0}^{n} B_{i,p}(u) \mathbf{v} \\
        &= C(u) + N_{i,p}(u) \mathbf{v}
    \end{align*}
\end{dem}
\paragraph{En pratique} Desormais, il nous est possible, sur la base de points de contrôle, d'un vecteur de noeud et du degré de la B-spline, de générer une courbe B-spline
comme ci-dessous:

\begin{figure}[H]
    \begin{tikzpicture}
        \Bezier{0,0}{1,1}{2,-1}{3,0}
        \begin{scope}[xshift=4cm]
            \Bezier{0,0}{9,2}{-2,2}{7,0}
        \end{scope}
        \begin{scope}[yshift=-5cm]
            \Bezier{0,0}{1,3}{2,3}{7,0}
    \end{scope}
    \begin{scope}[xshift=8cm,yshift=-5cm]
        \Bezier{0,0}{-2,4}{4,-1}{5,0}
    \end{scope}
\end{tikzpicture}
\caption{Courbes B-splines}
\label[figure]{fig:bsplines}
\end{figure}
\begin{lemme}
    Soit (U,)
    
\end{lemme}

\subsection{Morphing de courbes B-splines}
